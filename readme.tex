% Created 2022-09-05 Mon 19:16
\documentclass[9pt, b5paper]{article}
\usepackage{xeCJK}
\usepackage{minted}
\usepackage[T1]{fontenc}
\usepackage[scaled]{beraserif}
\usepackage[scaled]{berasans}
\usepackage[scaled]{beramono}
\usepackage{graphicx}
\usepackage{xcolor}
\usepackage{multirow}
\usepackage{multicol}
\usepackage{float}
\usepackage{textcomp}
\usepackage{algorithm}
\usepackage{algorithmic}
\usepackage{latexsym}
\usepackage{natbib}
\usepackage{geometry}
\geometry{left=1.2cm,right=1.2cm,top=1.5cm,bottom=1.2cm}
\newminted{common-lisp}{fontsize=\footnotesize} 
\usepackage[xetex,colorlinks=true,CJKbookmarks=true,linkcolor=blue,urlcolor=blue,menucolor=blue]{hyperref}
\author{deepwaterooo}
\date{\today}
\title{deepwaterooo deepwateroooMe -- I am the same GitHub account person}
\hypersetup{
  pdfkeywords={},
  pdfsubject={},
  pdfcreator={Emacs 27.1 (Org mode 8.2.7c)}}
\begin{document}

\maketitle
\tableofcontents


\section{要求}
\label{sec-1}
\begin{itemize}
\item Overview
\begin{itemize}
\item Build an \textbf{employee directory app} that shows a list of employees from the provided endpoint.
\item The app should display a list (or any kind of \textbf{collection view}!) which shows all the employees returned from the JSON endpoint described below.
\item Each item in the view should contain a \textbf{summary of the employee}, including their \textbf{photo, name, and team} at minimum. You may add more information to the summary if you want, or \textbf{sort employees in any fashion} you’d like – sort and group by name, team, etc.
\item There should be some UX to reload the employee list from within the app at any time. The UX can be done in any way you want: \textbf{a button, pull-to-refresh}, etc.
\item If there is any additional UI/UX you would like to add, feel free to do so! We only ask that you please \textbf{do not build any more screens} than this list. Do not worry about building custom controls or UI elements – using \textbf{system-provided, standard elements} is totally fine.
\item Be sure to \textbf{appropriately handle the normal variety of errors when querying an endpoint}. The app should \textbf{display useful loading, empty, and error states} where appropriate. \textbf{If images fail to load, displaying a placeholder} is fine.
\item One extra thing we ask is that you please ensure you \textbf{do not use more network bandwidth than necessary} – \textbf{load expensive resources such as photos on-demand only}.
\item The \textbf{employee list should not be persisted to disk}. You can reload it from the network \textbf{on each app launch and when refresh is requested} — but no more often than that unintentionally. (Android developers in particular should take care \textbf{not to make redundant network calls} when the \textbf{phone is rotated, or when memory is low}).
\item \textbf{Images}, however, should \textbf{be cached on disk} so as to not waste device bandwidth. You may use an \textbf{open source image caching solution}, or write your own caching. Do not rely upon HTTP caching for image caching.
\item Note that photos at a given URL will never change. Once one is loaded, you do not need to reload the photo. If an employee’s photo changes, they will be given a new photo URL.
\item Tests should be provided for the app. We do not expect 100\% code coverage, so please use your best judgment for what should be tested. We’re also interested only in unit tests. Feel free to skip snapshot or app tests.
\end{itemize}
\item MVVM: 需要数据驱动,viewModel里定义一个状态变量,来标记当前的活动状态
\begin{itemize}
\item If any employee is malformed, it is fine to invalidate the entire list of employees in the response - there is no need to exclude only malformed employees.
\item If there are no employees to show, the app should present an \textbf{empty state} view instead of an empty list.
\end{itemize}
\end{itemize}
\section{主要思路}
\label{sec-2}
\begin{itemize}
\item 这是一个看似要求极其简单,实则考验的知识点和深度有着相当的跨度的小项目。
\item 它们一定挑都要挑我出差到WSU的一个星期里来考验我,因为他们就是想要去打败一个人。呵呵,真正想要打败一个人,谈何容易,就凭这???
\item Retrofit + RxJava: 好像是更合适的,可以用注解,并且用得更为广泛
\begin{itemize}
\item 搜索关键字:Retrofit + OkHttp +RxJava 网络库构建
\item \textbf{OkHttp}: 网络请求处理,主要是在应用启动的时候,什么时机开始发布和调用网络请求。所以这个可以不用了,大家都喜欢新的更好用的库
\end{itemize}
\end{itemize}

\includegraphics[width=.9\linewidth]{./pic/readme_20220901_171033.png}
\begin{itemize}
\item \textbf{图片本地缓存}: 第三方库找一个,还是用AndroidX的Room
\begin{itemize}
\item 我 \textbf{现在数据库的问题} 是:我 \textbf{缓存保存了员工数据进数据库} ,但是这里说得很清楚了, \textbf{不用保存员工数据,只保存每个员工id所对应的图片就可以了}
\end{itemize}
\item 现在的难点:不知道怎么定义图片数据库,同时以OkHTTP respnose回来的连接起来
\item 应用的 \textbf{启动优化} :重中之重,需要借助这个小应用弄懂弄清楚, \textbf{不知道如何拆解网络请求的步骤,什么时候加载,初始化之类的?} 以达到较好的启动优化
\item 
\item \textbf{MVVM设计} :只有一个页面,相对就简单方便多了。工作中的案例是使用MVVM但自己编辑逻辑处理信号下发,与数据驱动的UI更新,没有实现双向数据绑定的;可是这里感觉 \textbf{双向数据绑定} 更简单,会有哪些可能的问题呢?这里基本可以当作不需要双向,因为一个UI按钮要求刷新是唯一的UI需求;更多的只是需要时候的数据往UI加载更新;所以 \textbf{可以简单使用观察者模式,UI观察数据的变化} 就可以了
\item \textbf{图片的加载与处理} :用样可以使用么第三方库 \textbf{glide}
\item \textbf{图片的加载与处理} :用样可以使用么第三方库 \textbf{CircularImageView}
\item \textbf{AndroidX RecyclerView} 的使用:选择相对更为高效和方便管理的库和数据结构来使用
\item \textbf{Constraint Layout vs Coordinate Layout}: 暂时先用任何简单的layout先能运行起一个大致的框架来,再进一步优化
\item 我丢掉了的文件呀,我写过的项目呀,不是在进Lucid之前写得好好的一个项目,现在源码全丢了。。。。。该死的GitHub\ldots{}..
\end{itemize}
% Emacs 27.1 (Org mode 8.2.7c)
\end{document}